% This file should be replaced with your file with an appendices (headings below are examples only)

% For compilation piecewise (see projekt.tex), it is necessary to uncomment it and change
%\documentclass[../projekt.tex]{subfiles}
%\begin{document}

% Placing of table of contents of the memory media here should be consulted with a supervisor
\chapter{Obsah priloženého pamäťového média}
\label{appendix:source}
Štruktúra odovzdaného archívu v~priloženej SD karte:

\setlength{\DTbaselineskip}{20pt}
\DTsetlength{0.5em}{1em}{0.3em}{0.4pt}{3pt}

\dirtree{%
.1 root.
.2 source/.
.3 ipfixcol2/
    \begin{minipage}[t]{10cm}
        \hspace{1em}
        \ldots
        \hspace{1em}
        \rmfamily Zdrojové súbory kolektoru \textit{IPFIXcol2}
    \end{minipage}.
.3 libfds/
    \begin{minipage}[t]{10cm}
        \hspace{2.5em}
        \ldots
        \hspace{1em}
        \rmfamily Zdrojové súbory knižnice \textit{libfds}
    \end{minipage}.
.3 README.md
    \begin{minipage}[t]{10cm}
        \hspace{1.47em}
        \ldots
        \hspace{1em}
        \rmfamily Popis kompilácie a spustenia kolektoru
    \end{minipage}.
.2 config/
    \begin{minipage}[t]{10cm}
        \hspace{4.21em}
        \ldots
        \hspace{1em}
        \rmfamily Príklady konfigurácií s~novými modulmi
    \end{minipage}.
.2 docs/
    \begin{minipage}[t]{10cm}
        \hspace{5.22em}
        \ldots
        \hspace{1em}
        \rmfamily Zdrojové texty tejto práce
    \end{minipage}.
.2 thesis.pdf
    \begin{minipage}[t]{10cm}
        \hspace{2.71em}
        \ldots
        \hspace{1em}
        \rmfamily Text bakalárskej práce
    \end{minipage}.
}

\chapter{Konfigurácia nových modulov}
\label{appendix:configuration}
Konfigurácia kolektoru je realizovaná pomocou jazyka XML, prostredníctvom ktorého je možné definovať vstupné, vnútorné a výstupné moduly kolektoru.
Príklady konfigurácií sú dostupné na stránke repozitára IPFIXcol2\footnote{\url{https://github.com/CESNET/ipfixcol2/tree/master/doc/data/configs}}.

\subsection*{Ukážkový obohacovací modul}

Ukážkový modul umožňuje definíciu položiek, ktorými sú obohatené záznamy prichádzajúcich IPFIX správ. Povinnou súčasťou definície položky je:
\begin{itemize}
    \setlength\itemsep{-0.5em}
    \item PEN (\textit{Private Enterprise Number})
    \item identifikačné číslo informačného elementu
    \item dátový typ (string, integer, decimal)
    \item hodnota položky
\end{itemize}

\noindent Voliteľne je možné definovať početnosť výskytu novej položky v~obohatenom zázname. Príklad konfigurácie ukážkového modulu pri jednorázovom obohacovaní tokov celočíselnou položkou:

\definecolor{darkBlue}{rgb}{0,0,0.6}

\lstdefinelanguage{myXML}{
    language=XML,
    keywordstyle=\color{darkBlue},
}

\lstset{language=myXML}
\begin{lstlisting}
<intermediate>
  <name>Dummy intermediate plugin</name>
  <plugin>dummy_enricher</plugin>
    <params>
        <field>
            <pen>0</pen>
            <id>21</id>
            <type>integer</type>
            <value>1023456</value>
            <times>1</times>
        </field>
    </params>
</intermediate>
\end{lstlisting} \newpage

\subsection*{Modul ASN}

Konfigurácia modulu ASN obsahuje ako jediný parameter cestu k~databázovému súboru \textit{GeoLite2-ASN}, ktorý bude využívaný v~priebehu obohacovania tokov. Príklad konfigurácie tohto modulu:

\begin{lstlisting}
<intermediate>
    <name>ASN enrichment</name>
    <plugin>asn</plugin>
    <params>
        <path>/home/user/maxmind/GeoLite2-ASN.mmdb</path>
    </params>
</intermediate>
\end{lstlisting}

\subsection*{Modul GeoIP}

Konfigurácia modulu GeoIP obsahuje parameter cesty k~databázovému súboru \textit{GeoLite2-City}. Okrem neho je možné definovať geolokačné položky, ktorými budú obohacované záznamy prijatých správ.
V~súčasnosti je podporované obohacovanie tokov nasledovnými položkami:

\begin{itemize}
    \setlength\itemsep{-0.5em}
    \item \textit{continent} -- textový reťazec označenia kontinentu (NA, EU, AF \ldots)
    \item \textit{country} -- označenie krajiny podľa normy ISO 3166-1 alpha-2 (SK, CZ, UK \ldots)
    \item \textit{city} -- názov mesta v~anglickom jazyku
    \item \textit{latitude} -- súradnice zemepisnej šírky
    \item \textit{longitude} -- súradnice zemepisnej dĺžky
\end{itemize}

Pri vynechaní definícií týchto položiek je použitá predvolená kombinácia obohatenia toku kódom krajiny a názvom mesta. Príklad konfigurácie modulu GeoIP s~položkami kódu kontinentu a názvu mesta:

\begin{lstlisting}
<intermediate>
    <name>GeoIP enrichment</name>
    <plugin>geoip</plugin>
      <params>
        <path>/home/user/maxmind/GeoLite2-City.mmdb</path>
        <fields>
            <field>continent</field>
            <field>city</field>
        </fields>
      </params>
</intermediate>
\end{lstlisting}