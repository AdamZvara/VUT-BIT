\documentclass{article}
\usepackage[a4paper, left=2cm, top=3cm, text={17cm, 24cm}]{geometry}
\usepackage[slovak]{babel}
\usepackage[utf8]{inputenc}
\usepackage[unicode]{hyperref}

\begin{document}

\begin{titlepage}
\begin{center}
    \Huge
    \textsc{Vysoké učení technické v Brně\\
    \huge
    Fakulta informačních technologií}\\
    \vspace{\stretch{0.382}}
    \LARGE
    Typografia a publikovanie\,--\,4. projekt\\
    \Huge
    Citácie a bibliografické odkazy\\
    \vspace{\stretch{0.618}}
    \Large
    \today \hfill Adam Zvara
\end{center}
\end{titlepage}

\section{Kybernetická bezpečnosť}

\subsection{Definícia}
Za kybernetickú bezpečnosť sa považuje kolekcia nástrojov, konceptov, usmernení a cvičení, ktorých cieľom je ochrana kybernetického prostredia organizácií a ich aktív \cite{Schatz2017}. Existuje niekoľko spôsobov, ktorými sa prejavuje kybernetický útok a taktiež niekoľko spôsobov, ako sa pred takýmito útokmi brániť.

\subsection{Modelovanie hrozieb}
Modelovanie hrozieb (\emph{Threat Modeling}) patrí medzi jeden zo základných spôsobov testovania bezpečnosti systému. Je založený na vytváraní abstraktných modelov, ktorých cieľom je odhaliť bezpečnostné problémy testovaného systému. Viac o tejto téme nájdete v \cite{Shostack2014}.

\section{Útoky}
Kybernetický útok (\emph{Cyber Attack}) je možné definovať ako akékoľvek úmyselné jednanie útočníka v kyberpriestore, ktoré smeruje proti záujmom inej osoby \cite{Kolouch2019}. Medzi základne druhý útokov patria phishing, spam, lámanie hesiel alebo DOS útoky.

Phishingom označujeme pokus o podvodné získanie súkromných informácií, ktorými môžu byť napríklad heslá alebo informácie o kreditných kartách, maskovaním sa za dôveryhodnú osobu pri elektronickej komunikácii. Viac o phishingu sa môžete dozvedieť na \cite{wiki:Phishing}.

Spamom \cite{wiki:Spam} sa rozumie nevyžiadané a hromadne rozosielanie správ identického obsahu veľkému množstvu príjemcov. Jedná sa o zneužívanie elektronickej komunikácie, najčastejšie e-mailu.

Lámanie hesiel (\emph{password cracking}) označuje rozličné metódy určené na odhalenie počítačových hesiel, ktoré sú  založené na systematickom výbere hesiel, pričom pre každé
vybraté heslo sa overuje jeho správnosť \cite{Hranicky2016}.

Útoky odoprenia služby (\emph{Denial Of Service}) sú útoky, ktorých účelom je zabrániť legitímnym užívateľom pristupovať k určitej službe. Existuje viacero variánt prevedenia DOS útoku, ktoré sú bližšie popísané v diplomovej práci \cite{Gerlichy2017}.

\section{Útočenie nie len na počítače}
\subsection{Sociálne inžinierstvo} 
Sociálne inžinierstvo predstavuje širokú škálu aktivít spojenú so získavaním citlivých informácií prostredníctvom medziľudských interakcií a psychologickej manipulácie. O všeobecnom postupe a opatreniach si môžete prečítať v članku \cite{SocialEngineering}.


\subsection{Kyberšikanovanie}
Kyberšikanovanie môžeme definovať \cite{CyberBullying} ako využívanie informačných technológií, najčastejšie sociálnych sietí, na úmyselné a  opakované šírenie nenávistného chovania jedinca alebo skupiny ľudí so zámerom poškodenia iných osôb.
Na túto tému existuje mnoho zaujímavých štúdií \cite{CyberBullyingStudy}, ktoré popisujú vnímanie kyberšikanovania samotnými študentmi.


\newpage
\bibliographystyle{czechiso}
\renewcommand{\refname}{Literatúra}
\bibliography{proj4}

\end{document}
